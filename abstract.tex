\begin{abstract}
The lower atmosphere has a rich variety of coherent structures, spanning from hundreds of meters to tens of kilometers. These are predominantly observable through cloud patterns, from the Earth's surface and space. Despite their substantial impact on Earth's energy and water balance, their driving forces remain inadequately understood.
 
The ATMOSCOPE project aims to revolutionize our understanding of the atmosphere by providing a comprehensive perspective on the diverse circulations within it. This initiative seeks to elucidate the intricate connections between the Earth's surface, the near-surface atmosphere, and the free atmosphere. To achieve this groundbreaking understanding, ATMOSCOPE comprises three phases: i) characterization:  utilizing ground-based and satellite measurements, informed by high-resolution 3D simulations, ATMOSCOPE will meticulously characterize mesoscale structures within the lower atmosphere, ii) understanding mechanisms: The project will explore the various mechanisms governing the formation, growth, and dissipation of these mesoscale structures, iii) synthesis: The project will synthesize findings from various mechanisms, seeking commonalities in circulations, and develop new theories to explain the impacts of these structures on atmospheric and surface coupling.
 
To support these intellectual pursuits, ATMOSCOPE will leverage three pivotal technical advancements, represented as the "lenses" of the project: i) cutting-edge simulation capabilities: state-of-the-art computational hardware will be harnessed to consolidate a virtual laboratory for in-depth study of coherent structures across the broad spectrum of scales on which they manifest, ii) ambitious field campaigns: employing innovative technologies and deploying exceptionally dense networks of in situ observations, ATMOSCOPE will directly and indirectly observe mesoscale structures, further enhancing our comprehension of their dynamics, iii) data assimilation methods: the project will develop data assimilation techniques to allow the integration of dense, time-based observations, including those collected during field campaigns, to enable a holistic system analysis and provide insights for future observations.
\end{abstract}
